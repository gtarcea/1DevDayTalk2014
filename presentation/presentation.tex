% Created 2014-11-07 Fri 11:14
\documentclass[presentation]{beamer}
\usepackage[utf8]{inputenc}
\usepackage[T1]{fontenc}
\usepackage{fixltx2e}
\usepackage{graphicx}
\usepackage{longtable}
\usepackage{float}
\usepackage{wrapfig}
\usepackage{rotating}
\usepackage[normalem]{ulem}
\usepackage{amsmath}
\usepackage{textcomp}
\usepackage{marvosym}
\usepackage{wasysym}
\usepackage{amssymb}
\usepackage{hyperref}
\tolerance=1000
\hypersetup{pdfauthor="V. Glenn Tarcea", pdftitle="Easy REST with AngularJS and Go", colorlinks, linkcolor=black, urlcolor=blue}
\usetheme{Hannover}
\usecolortheme[RGB={44, 92, 132}]{structure}
\author{V. Glenn Tarcea}
\date{November 2nd, 2014}
\title{Easy REST with AngularJS and Go}
\begin{document}

\maketitle
\begin{frame}{Outline}
\tableofcontents
\end{frame}


\section{Introduction}
\label{sec-1}

\begin{frame}[label=sec-1-1]{Speaker}
\begin{itemize}
\item Glenn Tarcea
\item Senior Developer at University of Michigan
\item Current Project: Materials Commons
\end{itemize}
\end{frame}

\begin{frame}[label=sec-1-2]{What this talk is about}
\end{frame}

\begin{frame}[label=sec-1-3]{What this talk doesn't cover}
\end{frame}

\section{AngularJS Setup}
\label{sec-2}
\begin{itemize}
\item How to setup angularjs
\begin{itemize}
\item A
\item B
\item C
\end{itemize}
\end{itemize}

\begin{frame}[fragile,label=sec-2-1]{AngularJS Setup - Imports}
 \begin{itemize}
\item We are going to import the required modules
\begin{itemize}
\item AngularJS router doesn't allow sub-views so we'll use ui-router
\item Restangular provides a nice REST interface
\end{itemize}
\item We don't technically need the extensions but they will make our lives easier
\end{itemize}
\begin{verbatim}
<script src=".../angularjs/1.3.1/angular.min.js">
</script>
<script src=".../angular-ui-router.min.0.2.11.js">
</script>
<script src=".../restangular.min.js">
</script>
\end{verbatim}
\end{frame}

\begin{frame}[fragile,label=sec-2-2]{AngularJS Setup - Setup our app}
 \begin{itemize}
\item To turn your app into an AngularJS app you need to add ng-app.
\item Here we set up a name of our name. We'll see more about this.
\end{itemize}

\begin{verbatim}
<html ng-app="myapp" lang="en">
  <head>...</head>
  <body>
\end{verbatim}
\end{frame}

\begin{frame}[fragile,label=sec-2-3]{AngularJS Setup - View}
 \begin{itemize}
\item ui-view is where we'll load page content.
\item ui-router allows sub views. Basically we can
have a tree of views and states.
\end{itemize}

\begin{verbatim}
    <div class="main-content">
      <!-- Setup location for our main view -->
      <div ui-view>
      </div>
    </div>
  </body>
</html>
\end{verbatim}
\end{frame}

\begin{frame}[fragile,label=sec-2-4]{AngularJS Setup - Putting it all together}
 \begin{itemize}
\item So here is what our index.html html looks like
\end{itemize}
\begin{verbatim}
<html ng-app="myapp" lang="en">
  <head>...</head>
  <body>
    <div class="main-content">
      <div ui-view>
      </div>
    </div>
    <script>...</script>
  </body>
</html>
\end{verbatim}
\end{frame}

\section{Configure App}
\label{sec-3}
\begin{itemize}
\item To configure our App we need to set up our routes and module references.
\begin{itemize}
\item Routes control which pages to display
\item Module references give us an easy way to reference the different pieces of our project
\begin{itemize}
\item Controllers
\item Filters
\item Services
\item Directives
\end{itemize}
\end{itemize}
\end{itemize}

\begin{frame}[fragile,label=sec-3-1]{Module References}
 \begin{itemize}
\item Set references to our app modules.
\begin{itemize}
\item We break our app into different modules for the models in AngularJS.
\end{itemize}
\end{itemize}
\begin{verbatim}
var App = App || {};
App.Services = angular.module('app.services', []);
App.Controllers = angular.module('app.cntrlrs', []);
App.Filters = angular.module('app.filters', []);
App.Directives = angular.module('app.directives', []);
var app = angular.module('myapp', [
    "ui.router", "restangular",
    "app.services", "app.cntrlrs", "app.filters",
    "app.directives"
]);
\end{verbatim}
\end{frame}

\begin{frame}[fragile,label=sec-3-2]{Interlude: Dependency Injection}
 \begin{itemize}
\item AngularJS makes extensive use of dependency injection
\item It does inject based on the name
\begin{itemize}
\item This doesn't work when minimizing your code
\end{itemize}
\item You have 2 options when you want to minimize
\begin{itemize}
\item You can use a plugin that will rewrite your code
\item Or you can write your code so it can be minimized
\begin{itemize}
\item I use this option throughout the example code
\end{itemize}
\end{itemize}
\end{itemize}
\begin{verbatim}
App.Controllers.controller("name-of-controller",
                       ["dependency1Name", "...",
                        controllerFunction]);
function controllerFunction (dependency1Name) {
    // ...
}
\end{verbatim}
\end{frame}

\begin{frame}[fragile,label=sec-3-3]{Configure our Routes}
 \begin{itemize}
\item We set up 2 routes and a default route
\end{itemize}
\begin{verbatim}
app.config(["$stateProvider", "$urlRouterProvider", appConfig]);
function appConfig($stateProvider, $urlRouterProvider) {
    $stateProvider
        .state("users", {
            url: "/users",
            templateUrl: "app/users.html",
            controller: "usersController"
        })
        .state("users.add", {
            url: "/add",
            templateUrl: "app/add.html",
            controller: "addUserController"
        });
    $urlRouterProvider.otherwise("/users");
}
\end{verbatim}
\end{frame}

\section{Views}
\label{sec-4}

\section{REST using Restangular}
\label{sec-5}

\section{Go Setup}
\label{sec-6}
\begin{itemize}
\item Now well configure a Go server
\item We'll use this server for our REST services and to serve our web pages
\begin{itemize}
\item Go has an HTTP interface that makes writing web servers and services very easy
\begin{itemize}
\item This is one of the nicest pieces of using Go
\end{itemize}
\end{itemize}
\end{itemize}

\begin{frame}[fragile,label=sec-6-1]{Go Web Server Setup}
 \begin{itemize}
\item We'll point our web server at our apps directory
\item This will be our default route
\begin{itemize}
\item The server will automatically pick up the index.html file
\end{itemize}
\end{itemize}
\begin{verbatim}
webdir := ...
dir := http.Dir(webdir)
http.Handle("/", http.FileServer(dir))
addr := "localhost:8081"
fmt.Println(http.ListenAndServe(addr, nil))
\end{verbatim}
\end{frame}

\begin{frame}[fragile,label=sec-6-2]{REST Setup}
 \begin{itemize}
\item We'll use a nice REST extension package: go-restful
\item Because this package uses HTTP interfaces we can use standard Go http to setup
\end{itemize}
\begin{verbatim}
container := ...

// All REST calls come through a /api/... route.
// We strip off /api before sending on to our
// container this way the container doesn't
// care about the prefix.
http.Handle("/api/", http.StripPrefix("/api",
        container))
\end{verbatim}
\end{frame}

\section{Go REST Service}
\label{sec-7}
\begin{verbatim}
ws := new(restful.WebService)
ws.Path("/users").
        Consumes(restful.MIME_JSON).
        Produces(restful.MIME_JSON)

ws.Route(ws.GET("").To(rest.RouteHandler(r.getAllUsers)).
        Doc("Retrieves all users").
        Writes([]schema.User{}))
\end{verbatim}

\begin{frame}[fragile,label=sec-7-1]{Service Implementation}
 \begin{verbatim}
func (r *usersResource) createUser(request *restful.Request,
        response *restful.Response, user schema.User) (error, interface{}) {

        var req userReq
        if err := request.ReadEntity(&req); err != nil {
                return err, nil
        }
        u, err := r.users.CreateUser(req.Email, req.Fullname)
        return err, u
}
\end{verbatim}
\end{frame}
% Emacs 24.3.1 (Org mode 8.2.6)
\end{document}
