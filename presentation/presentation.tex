% Created 2014-11-03 Mon 11:36
\documentclass[presentation]{beamer}
\usepackage[utf8]{inputenc}
\usepackage[T1]{fontenc}
\usepackage{fixltx2e}
\usepackage{graphicx}
\usepackage{longtable}
\usepackage{float}
\usepackage{wrapfig}
\usepackage{rotating}
\usepackage[normalem]{ulem}
\usepackage{amsmath}
\usepackage{textcomp}
\usepackage{marvosym}
\usepackage{wasysym}
\usepackage{amssymb}
\usepackage{hyperref}
\tolerance=1000
\hypersetup{pdfauthor="V. Glenn Tarcea", pdftitle="Easy REST with AngularJS and Go", colorlinks, linkcolor=black, urlcolor=blue}
\usetheme{Hannover}
\usecolortheme[RGB={44, 92, 132}]{structure}
\author{V. Glenn Tarcea}
\date{November 2nd, 2014}
\title{Easy REST with AngularJS and Go}
\begin{document}

\maketitle
\begin{frame}{Outline}
\tableofcontents
\end{frame}


\section{Introduction}
\label{sec-1}

\begin{frame}[label=sec-1-1]{Speaker}
\begin{itemize}
\item Glenn Tarcea
\item Senior Developer at University of Michigan
\item Current Project: Materials Commons
\end{itemize}
\end{frame}

\begin{frame}[label=sec-1-2]{What this talk is about}
\end{frame}

\begin{frame}[label=sec-1-3]{What this talk doesn't cover}
\end{frame}

\section{AngularJS Setup}
\label{sec-2}
\begin{itemize}
\item How to setup angularjs
\end{itemize}

\begin{frame}[fragile,label=sec-2-1]{AngularJS Setup - Imports}
 \begin{itemize}
\item We are going to import the required modules
\begin{itemize}
\item AngularJS router doesn't allow sub-views so we'll use ui-router
\item Restangular provides a nice REST interface
\end{itemize}
\item We don't technically need the extensions but they will make our lives easier
\end{itemize}
\begin{verbatim}
<script src=".../angularjs/1.3.1/angular.min.js">
</script>
<script src=".../angular-ui-router.min.0.2.11.js">
</script>
<script src=".../restangular.min.js">
</script>
\end{verbatim}

\begin{frame}[fragile,label=sec-2-2]{AngularJS Setup - Setup our app}
 \begin{verbatim}
<html ng-app="myapp" lang="en">
  <head>...</head>
  <body>
\end{verbatim}
\end{frame}

\begin{frame}[fragile,label=sec-2-3]{AngularJS Setup - View}
 \begin{verbatim}
    <div class="main-content">
      <!-- Setup location for our main view -->
      <div ui-view>
      </div>
    </div>
  </body>
</html>
\end{verbatim}
\end{frame}

\begin{frame}[fragile,label=sec-2-4]{AngularJS Setup - Putting it all together}
 \begin{verbatim}
<html ng-app="myapp" lang="en">
  <head>...</head>
  <body>
    <div class="main-content">
      <div ui-view>
      </div>
    </div>
    <script>...</script>
  </body>
</html>
\end{verbatim}
\end{frame}

\begin{frame}[fragile,label=sec-2-5]{Configure our App}
 \begin{verbatim}
var App = App || {};

App.Constants = angular.module('app.constants', []);
App.Services = angular.module('app.services', []);
App.Controllers = angular.module('app.controllers', []);
App.Filters = angular.module('app.filters', []);
App.Directives = angular.module('app.directives', []);
var app = angular.module('myapp', [
    "ui.router", "restangular",
    "app.constants", "app.services",
    "app.controllers", "app.filters",
    "app.directives"
]);
\end{verbatim}
\end{frame}

\begin{frame}[fragile,label=sec-2-6]{Configure our Routes}
 \begin{verbatim}
app.config(["$stateProvider", appConfig]);
function appConfig($stateProvider) {
    $stateProvider
        .state("users", {
            url: "/users",
            templateUrl: "app/users.html",
            controller: "usersController"
        })
        .state("users.add", {
            url: "/add",
            templateUrl: "app/add.html",
            controller: "addUserController"
        });
}
\end{verbatim}
\end{frame}
\end{frame}
% Emacs 24.3.1 (Org mode 8.2.6)
\end{document}
