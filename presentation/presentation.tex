% Created 2014-11-12 Wed 21:34
\documentclass[presentation]{beamer}
\usepackage[utf8]{inputenc}
\usepackage[T1]{fontenc}
\usepackage{fixltx2e}
\usepackage{graphicx}
\usepackage{longtable}
\usepackage{float}
\usepackage{wrapfig}
\usepackage{rotating}
\usepackage[normalem]{ulem}
\usepackage{amsmath}
\usepackage{textcomp}
\usepackage{marvosym}
\usepackage{wasysym}
\usepackage{amssymb}
\usepackage{hyperref}
\tolerance=1000
\hypersetup{pdfauthor="V. Glenn Tarcea", pdftitle="Easy REST with AngularJS and Go", colorlinks, linkcolor=black, urlcolor=blue}
\usetheme{Hannover}
\usecolortheme[RGB={44, 92, 132}]{structure}
\author{V. Glenn Tarcea}
\date{November 15th, 2014}
\title{AngularJS and Go}
\begin{document}

\maketitle
\begin{frame}{Outline}
\tableofcontents
\end{frame}


\section{Introduction}
\label{sec-1}

\begin{frame}[label=sec-1-1]{About Me}
\begin{itemize}
\item Glenn Tarcea
\item Senior Developer at University of Michigan
\item Current Project: Materials Commons
\end{itemize}
\end{frame}

\begin{frame}[label=sec-1-2]{Materials Commons}
\begin{itemize}
\item Materials Commons is an online collaborative space for Metals Researchers
\item We have open sourced all the code for Materials Commons:
\begin{itemize}
\item Go, Javascript, Java, Python, Erlang, C
\end{itemize}
\item You can find our code at:
\begin{itemize}
\item \url{https://github.com/materials-commons}
\item \url{https://github.com/prisms-center/materialscommons.org}
\end{itemize}
\item There are alot of nice (if sometimes a bit rough) packages:
\begin{itemize}
\item Erlang: gen stomp, resource discovery, process monitoring, OS interfaces
\item Go: Utilities, config, file transfer, FlowJS server
\item Javascript: AngularStomp
\item Java: DM3 Parser for Tika (not touched in a while)
\end{itemize}
\end{itemize}
\end{frame}

\begin{frame}[label=sec-1-3]{What this talk is about}
\begin{itemize}
\item This talk will cover creating a website using
\begin{itemize}
\item Go and AngularJS
\item Websockets
\item REST
\item JWT
\end{itemize}
\item The site will allow for simple "collaboration"
\begin{itemize}
\item By using broadcasts to keep each site in sync
\end{itemize}
\end{itemize}
\end{frame}

\begin{frame}[label=sec-1-4]{What this talk doesn't cover}
\begin{itemize}
\item This talk is not a Go or AngularJS tutorial
\begin{itemize}
\item We will go over some aspects of both but will not spend a lot of time on the basics
\end{itemize}
\item It won't cover all aspects of the application
\begin{itemize}
\item We will elide some details but you can refer to the sample app to get all the details
\end{itemize}
\end{itemize}
\end{frame}

\begin{frame}[label=sec-1-5]{Where to get the app}
\begin{itemize}
\item I've set up a Github repo that contains the working app
\begin{itemize}
\item \url{https://github.com/gtarcea/1DevDayTalk2014}
\end{itemize}
\item The README.org goes over getting it running
\begin{itemize}
\item In a nutshell:
\begin{itemize}
\item Install go
\item Install godep (go get github.com/tools/godep)
\item make run
\end{itemize}
\end{itemize}
\item The intent of this app is to give you a nice starting point
\begin{itemize}
\item It gives you a working JWT, Websocket, REST based application
\item With client side authentication
\item Reconnect
\item Broadcast to keep all connected clients updated
\end{itemize}
\item It looks simple but there is a lot going on
\end{itemize}
\end{frame}

\section{AngularJS Setup}
\label{sec-2}

\begin{frame}[label=sec-2-1]{Overview}
\begin{itemize}
\item We'll cover the basics of setting up an angular app and configuring the needed packages
\item We use a few client libraries to make our lives easier
\begin{itemize}
\item ui-router to give us multiple state based routes
\item ng-websocket for websocket communication
\item angular-jwt for easy JWT integration
\item Restangular for REST communication
\end{itemize}
\item We will cover configuring and integrating these packages
\end{itemize}
\end{frame}

\begin{frame}[fragile,label=sec-2-2]{AngularJS Setup - Setup our app}
 \begin{itemize}
\item To turn your app into an AngularJS app you need to add ng-app.
\item Here we set up a name of our name. We'll see more about this.
\end{itemize}

\begin{verbatim}
<html ng-app="myapp" lang="en">
  <head>...</head>
  <body>
\end{verbatim}
\end{frame}

\begin{frame}[fragile,label=sec-2-3]{AngularJS Setup - View}
 \begin{itemize}
\item ui-view is where we'll load page content.
\item ui-router allows sub views. Basically we can
have a tree of views and states.
\end{itemize}

\begin{verbatim}
    <div class="main-content">
      <!-- Setup location for our main view -->
      <div ui-view>
      </div>
    </div>
  </body>
</html>
\end{verbatim}
\end{frame}

\begin{frame}[fragile,label=sec-2-4]{AngularJS Setup - Putting it all together}
 \begin{itemize}
\item So here is what our index.html html looks like
\end{itemize}
\begin{verbatim}
<html ng-app="myapp" lang="en">
  <head>...</head>
  <body>
    <div class="main-content">
      <div ui-view>
      </div>
    </div>
    <script>...</script>
  </body>
</html>
\end{verbatim}
\end{frame}

\section{Configure App}
\label{sec-3}

\begin{frame}[label=sec-3-1]{Overview}
\begin{itemize}
\item To configure our App we need to set up our routes and module references.
\begin{itemize}
\item Routes control which pages to display
\item Module references give us an easy way to reference the different pieces of our project
\begin{itemize}
\item Controllers
\item Filters
\item Services
\item Directives
\end{itemize}
\end{itemize}
\end{itemize}
\end{frame}

\begin{frame}[fragile,label=sec-3-2]{Module References}
 \begin{itemize}
\item Set references to our app modules.
\begin{itemize}
\item We break our app into different modules for the models in AngularJS.
\end{itemize}
\end{itemize}
\begin{verbatim}
var App = App || {};
App.Services = angular.module('app.services', []);
App.Controllers = angular.module('app.cntrlrs', []);
App.Filters = angular.module('app.filters', []);
App.Directives = angular.module('app.directives', []);
var app = angular.module('myapp', [
    "ui.router", "restangular",
    "app.services", "app.cntrlrs", "app.filters",
    "app.directives"
]);
\end{verbatim}
\end{frame}

\begin{frame}[fragile,label=sec-3-3]{Configure our Routes}
 \begin{itemize}
\item We set up 2 routes and a default route
\end{itemize}
\begin{verbatim}
app.config(["$stateProvider", "$urlRouterProvider",
            appConfig]);
function appConfig($stateProvider, $urlRouterProvider) {
    $stateProvider
        .state("users", {
            url: "/users",
            templateUrl: "app/users.html",
            controller: "usersController"
        })
        .state("users.add", {
            url: "/add",
            templateUrl: "app/add.html",
            controller: "addUserController"
        });
    $urlRouterProvider.otherwise("/users");
}
\end{verbatim}
\end{frame}

\begin{frame}[label=sec-3-4]{Configure Authentication}
\begin{itemize}
\item To configure authentication we need to
\begin{itemize}
\item Control access to protected areas of our app
\item Track user authentication
\item Setup JWT Headers for all REST calls
\end{itemize}
\end{itemize}
\end{frame}
\begin{frame}[fragile,shrink=10,label=sec-3-5]{Controlling Access}
 \begin{verbatim}
app.run(["$rootScope", "User", "$state", appRun]);
function appRun($rootScope, User, $state) {
    // $stateChangeStart is fired when a route change
    // is starting. Here we check if the user is already
    // authenticatd. If they aren't then we redirect
    // them to the login page.
    $rootScope.$on('$stateChangeStart', stateChange);

    function stateChange(event, toState, toParams) {
        if (!User.isAuthenticated()) {
            if (toState.url !== "/login") {
                // Cancel whatever route we were going
                // to and instead go to the login page.
                event.preventDefault();
                $state.go("login");
            }
        }
    }
}
\end{verbatim}
\end{frame}

\begin{frame}[fragile,shrink=10,label=sec-3-6]{Configuring JWT}
 \begin{itemize}
\item The following code is also in appConfig (where we also configured the routes)
\end{itemize}
\begin{verbatim}
// The JWT token is stored in sessionStorage. When our
// app starts up we explicitly clear the previous token.
sessionStorage.setItem("token", null);

// This interceptor will set the Authorization field
// in the header with the JWT token.
jwtInterceptorProvider.tokenGetter = function() {
    var token = sessionStorage.getItem("token");
    return token ? token : "";
};
$httpProvider.interceptors.push("jwtInterceptor");
\end{verbatim}
\end{frame}

\begin{frame}[label=sec-3-7]{Configure Websockets}
\end{frame}

\section{Views}
\label{sec-4}
\begin{frame}[label=sec-4-1]{Overview}
\end{frame}
\section{REST using Restangular}
\label{sec-5}
\begin{frame}[label=sec-5-1]{Overview}
\end{frame}
\section{Go Setup}
\label{sec-6}
\begin{frame}[label=sec-6-1]{Overview}
\begin{itemize}
\item Now well configure a Go server
\item We'll use this server for our REST services and to serve our web pages
\begin{itemize}
\item Go has an HTTP interface that makes writing web servers and services very easy
\begin{itemize}
\item This is one of the nicest pieces of using Go
\end{itemize}
\end{itemize}
\end{itemize}
\end{frame}

\begin{frame}[fragile,label=sec-6-2]{Go Web Server Setup}
 \begin{itemize}
\item We'll point our web server at our apps directory
\item This will be our default route
\begin{itemize}
\item The server will automatically pick up the index.html file
\end{itemize}
\end{itemize}
\begin{verbatim}
webdir := ...
dir := http.Dir(webdir)
http.Handle("/", http.FileServer(dir))
addr := "localhost:8081"
fmt.Println(http.ListenAndServe(addr, nil))
\end{verbatim}
\end{frame}

\begin{frame}[fragile,label=sec-6-3]{REST Setup}
 \begin{itemize}
\item We'll use a nice REST extension package: go-restful
\item Because this package uses HTTP interfaces we can use standard Go http to setup
\end{itemize}
\begin{verbatim}
container := ...

// All REST calls come through a /api/... route.
// We strip off /api before sending on to our
// container this way the container doesn't
// care about the prefix.
http.Handle("/api/", http.StripPrefix("/api",
        container))
\end{verbatim}
\end{frame}

\section{Go REST Service}
\label{sec-7}
\begin{frame}[fragile,label=sec-7-1]{Overview}
 \begin{verbatim}
ws := new(restful.WebService)
ws.Path("/users").
        Consumes(restful.MIME_JSON).
        Produces(restful.MIME_JSON)

ws.Route(ws.GET("").To(rest.RouteHandler(r.getAllUsers)).
        Doc("Retrieves all users").
        Writes([]schema.User{}))
\end{verbatim}
\end{frame}

\begin{frame}[fragile,label=sec-7-2]{Service Implementation}
 \begin{verbatim}
func (r *usersResource) createUser(request *restful.Request,
        response *restful.Response, user schema.User) (error, interface{}) {

        var req userReq
        if err := request.ReadEntity(&req); err != nil {
                return err, nil
        }
        u, err := r.users.CreateUser(req.Email, req.Fullname)
        return err, u
}
\end{verbatim}
\end{frame}
% Emacs 24.3.1 (Org mode 8.2.6)
\end{document}
